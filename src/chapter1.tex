

\chapterimage{vgtu3.jpg} % Chapter heading image

\chapter{Introduction to lambda calculus}

\section{Definition of function}\index{Definition of function}

There are 2 ways to describe functions: extensional and intensional. Extensional became popular with set theory. 
In set theory relation between 2 sets is defined as Cartesian product: $R \subseteq X \times Y$. Function
is a special kind of relation:
$X$ is called function's domain and $Y$ is codomain, and function $f : X \rightarrow Y$ 
is a set of pairs $f \subseteq X \times Y$ such that for each $x \in X$, there exists exactly one 
$y \in Y$ such that $(x, y) \in f$ . Two functions $f, g : X \rightarrow Y$ are
considered equal if they yield the same output on each input, i.e., $f(x) = g(x)$ for all $x \in X$.
\\
As we will see later, this is a very convienent way to compare functions or express functions with when domain cardinality is low.
In the early 17th century Blaise Pascal built automation device for arithmetical functions. Gottfried Leibniz improved it
to perform automated multiplication and division in the end of 17th century. He first came to idea of machine that could 
calculate mathematical statements and give true or false answers. This started to be known as $Entscheidungsproblem$ - a decision
problem. In 1928, logician David Hilbert formulated it officially. He believed that, by using automation, any problem can be solved.
Just before first attempts of calculation automation were finally successful,
\footnote{first artual attempt to build automated generic purpose machine was in 1833, when Charles Babbage, Augusta Ada Byron and 
others were working on the first generic purpose machine - Analytical Engine}
around 1940's, Alan Turing and Alonzo Church independently proved that $Entscheidungsproblem$ is impossible. It was clear, that
computers or, generally speaking - algorithms cannot solve even (relatively) simple task - answer will some function return result
or loop forever.
\\
Algorithm mentioned here is a way to define functions intensionally.

\subsection{Definition of lambda}\index{Definition of lambda}

Lambda calculus is expressed as context free grammar in following BNF form:

\[ e ::= x \ | \ \lambda{x}.{e} \ | \  e_0~e_1 \]

$x$ here is a variable, there is unlimited number of variables - we can invent them as we need. Second one - \lam${x}.{e}$ 
is called lambda abstraction and denotes function. Third one - $e_0~e_§$ is an application and denotes "invoking" a function ($e_0$)
by sending it an argument ($e1$)
Since BNF is recursive definition, we can quickly spot that if lambda abstraction is a function, we can apply it another function.
This defines one of the essential lambda calculus properties - function is an element of the language. In imperative language we use
language constructs to create functions and they become part of the solution we implement. In lambda language we can construct 
functions by composing another functions - they are part of the language.
\\
Though lambda is a model to describe algorithms, we started to talk it as a language. Indeed, lambda is also a programming language,
which, adeed some augmentations is able to describe concepts of problem solving. Functional languages like Haskell, Ocaml, Miranda and others
have augmented lambda language and evolved to become fully functional and powerful programming languages.
\\
Definition above is very terse and we need rules on how to operate lambda structures.

\subsection{Notation rules}

To minimize number of parentheses, following conventions used:
\\Application assiociates to the left. If there are more than 2 members in application, consider application of first 2 on the left, then their
result is applied to third member, etc.:

\[e_0\ e_1\ e_2\ ::=\ ((e_0\ e_1)\ e_2)\]

Same with more elements in application:

\[e_0\ e_1\ e_2\ e_3\ ::=\ (((e_0\ e_1)\ e_2)\ e_3)\]

Lambda scope extends to the right as far as possible:

\[\lambda{x}.e_0\ e_1\ e_2 ... e_n ::= (\lambda{x}.(e_0\ e_1\ e_2 ... e_n ))\]

This scope maybe "ended" by parentheses: scope of $x$ does not extends to second expression $\lambda{y}...$

\[(\lambda{x}.e_0\ e_1 ... e_n\ )\ (\lambda{y}.f_0\ f_1\ ...\ f_n)\]

Rightmost lambda binds closest expression:

\[\lambda{xyz}.e\ ::= \ (\lambda{x}.(\lambda{y}.(\lambda{z}.e))) \]

Application still has priority:

\[ \lambda{xy}.eyz ::= (\lambda{x}.(\lambda{y}.((ey)\ z))) \]

\subsection{Variable Binding}
A lambda expression introduces a $binding\ occurrence$ - all variables in lambda abstraction before the dot. 
Variable $x$ is $free$ if it does not occur within scope of \lam$x$.
This is called $lexical\ scoping$; the variable's scope is
defined by the text (body) of the program. It is ``lexical'', because it is
possible to determine its scope in the text, in lexemas of the language as soon as we have 
the body of a program, before it's execution.

%%TODO - examples

Below red circles denote bound variables and blue ones denote free:

\[ (\lambda{x}.\circleblue{y}\ \circlered{x}) (\lambda{y}.\circlered{y}\ \circleblue{x}) \]
\[ \lambda{xz}.\circleblue{y}\ \circlered{x} (\lambda{y}.\circlered{y}\ \circlered{z}) \]

\subsection{Free and bound variables}
Roughly, variable is free if it does not occur in lambda's abstraction binding expression. Following table shows basic cases of
free and bound variables

\begin{tabular}{l c c p{7cm}}
  E & Case &  Does x occur free in E? & Explanation \\
  \hline
  x & 1 & yes & x appears in (is equal to) E and E does not contain any binding occurrences (no $\lambda$) \\
  y & 1 & no  & x does not occur in E and thus cannot occur free in it \\
  (x y) & 2 & yes & x occurs free in the first component of the function application (recursive application of case 1) \\
  (y x) & 2 & yes & x occurs free in the second component of the function application (recursive application of case 1) \\
  (y z) & 2 & no  & x occurs free in neither the first nor the second component of the function application (doubly recursive application of case 1) \\
  $\lambda$$z.x$  & 3 & yes & x is different from z (the parameter of the lambda abstraction) and x occurs free in the body of the lambda abstraction (recursive application of case 1). Note that the body is what is left of the lambda abstraction after the binding occurrence (i.e., $\lambda$$z.$) is removed \\
  $\lambda$$z.z$  & 3 & no  & x is different from z (the parameter of the lambda abstraction) and x does not occur (at all, and thus not free either) in the body of the lambda abstraction \\
  $\lambda$$z.\lambda$$x.x$ & 3 & no & x is different from z (the parameter of the lambda abstraction) but x does not occur free in the body of the lambda abstraction (recursive application of case 3). Note that the body in this case is the lambda abstraction $\lambda$$x.x$ \\
  \lam$x.y$ or \lam$x.x$ & 3 & no & x is identical to the parameter of the lambda abstraction E. x cannot be free in E since any free occurrences of x in the body of E would become bound in E by the leading binding occurrence of x 
\end{tabular}


\subsection{Reductions}
Reductions are rules how we process lambda expressions.
%% todo alpha reduction
%% todo beta reduction

\subsection{Combinators}
Expressions containing free variables, f.ex like $x$ or $(xy)$ have not much meaning if variables are plain (no context). These expressions cannot be reduced, hence
no application is possible. What would be more meaningful (in term of reductions) if expressions would have no free variables, which would guarantee that 
lambda expression applied to arguments will reduce.
\\
Expressions with no free variables are called \textbf{combinators}.

\subsection{Pairs and lists}
Pair is a combinator that will result in an expression, "capturing" 2 values:
\[pair ::= \lambda{abf}.fab \]
To recap - we use "sugared" version where lambda associates to the right until dot and applications associates to the right. 
This effectively means:
\[pair ::= \lambda{a}.\lambda{b}.\lambda{f}.((f a)\ b) \]

by applying 2 arguments to pair combinator we will get expression, f.ex:
%% todo a sample how to create a pair

to extract the values we will use the following expressions:
\[first ::= \lambda{ab}.a \] 
\[second ::= \lambda{ab}.b \] 

